\documentclass{article}
\usepackage[a4paper,top=1cm,bottom=1cm,left=1cm,right=1cm,marginparwidth=1.75cm]{geometry}
\usepackage[numbered,framed]{matlab-prettifier}
\usepackage{xcolor}
\usepackage{listings}
\usepackage{multicol}
\setlength\columnsep{1cm}

\definecolor{mGreen}{rgb}{0,0.6,0}
\definecolor{mGray}{rgb}{0.5,0.5,0.5}
\definecolor{mPurple}{rgb}{0.58,0,0.82}
\definecolor{backgroundColour}{rgb}{0.95,0.95,0.92}

\lstdefinestyle{CStyle}{
    backgroundcolor=\color{backgroundColour},   
    commentstyle=\color{mGreen},
    keywordstyle=\color{magenta},
    numberstyle=\tiny\color{mGray},
    stringstyle=\color{mPurple},
    basicstyle=\footnotesize,
    breakatwhitespace=false,         
    breaklines=true,                 
    captionpos=b,                    
    keepspaces=true,                 
    numbers=left,                    
    numbersep=5pt,                  
    showspaces=false,                
    showstringspaces=false,
    showtabs=false,                  
    tabsize=2,
    language=C
}

\begin{document}

\section{MATLAB}

\subsection{User I/O and data types}
\begin{lstlisting}[style=Matlab-editor]
prompt = "Enter a value: ";
x = input(prompt);
txt = input("Enter some text: ", "s");

formatSpec = '%s, %.3f doubled is %.3d';
fprintf(formatSpec, txt, x, 2*double(x))
\end{lstlisting}

\subsection{Loops}
\begin{lstlisting}[style=Matlab-editor]
for index = initVal:step:endVal
   statements
   break
end
while a == b || ~(b > c && c <= d)
    statements
end
\end{lstlisting}

\subsection{Matrices and vectors}
\begin{lstlisting}[style=Matlab-editor]
mtr = zeros(2,4);
vct = [1,2,3,4,5,6,7,8];

mtr(1,1) = vct(8);
\end{lstlisting}

\subsection{Conditional execution}
\begin{multicols}{2}
\begin{lstlisting}[style=Matlab-editor]
if expression
    statements
elseif expression
    statements
else
    statements
end
\end{lstlisting}\columnbreak
\begin{lstlisting}[style=Matlab-editor]
switch switch_expression
   case case_expression
      statements
   case case_expression
      statements
    ...
   otherwise
      statements
end
\end{lstlisting}
\end{multicols}


\subsection{Custom functions}
\begin{lstlisting}[style=Matlab-editor]
function [a,b] = ftn(a,b)
    statements
end

[a,b] = ftn(a,b)
\end{lstlisting}

\subsection{Other functions}
\begin{lstlisting}[style=Matlab-editor]
num = abs(num);
num = mod(num,2);
tf = strcmp(s1, s2)

for i = 1:size(mtr,1)
    for j = 1:size(mtr,2)
        statements
    end
end
\end{lstlisting}
\pagebreak

\section{C}

\subsection{User I/O}
\begin{lstlisting}[style=CStyle]
int main(void) {
  int n;
  printf("Enter a number: "); // requires stdio.h
  scanf("%d", &n); // requires stdio.h
  printf("The number is %d.\n", n);
  return 0;
}
\end{lstlisting}

\subsection{Loops}
\begin{lstlisting}[style=CStyle]
  for (int i = 0; i < 10; i++) {
    printf("%d\n", i);
  }
  
  int j = 0;
  while (j < 10) {
    printf("%d\n", j);
    j++;
  }
\end{lstlisting}

\subsection{Matrices}
\begin{lstlisting}[style=CStyle]
  int matrix[3][3] = {{1, 2, 3}, {4, 5, 6}, {7, 8, 9}};
  
  int *matrix2 = malloc(sizeof(int) * 2 * 2); // requires stdlib.h
  matrix2[0] = 1;
  matrix2[1] = 2;
  matrix2[2] = 3;
  matrix2[3] = 4;
  free(matrix2);
\end{lstlisting}

\subsection{Conditional execution}
\begin{multicols}{2}
\begin{lstlisting}[style=CStyle]
  int a = 1;
  if (a == 1) {
    printf("a is 1\n");
  } else if (a == 2) {
    printf("a is 2\n");
  } else {
    printf("a is neither 1 nor 2\n");
  }
\end{lstlisting}\columnbreak
\begin{lstlisting}[style=CStyle]
  int b = 1;
  switch (b) {
    case 1:
      printf("b is 1\n");
      break;
    case 2:
      printf("b is 2\n");
      break;
    default:
      printf("b is neither 1 nor 2\n");
      break;
  }
\end{lstlisting}
\end{multicols}

\subsection{Custom function and rng}
\begin{lstlisting}[style=CStyle]
int rand_num(int min, int max) {
  srand(time(NULL)); // requires time.h
  return min + (rand() % (max - min + 1)); // requires stdlib.h
}
\end{lstlisting}

\subsection{String functions}
\begin{lstlisting}[style=CStyle]
// requires string.h
  strcat(dest, src); // concatenate src to end of dest
  strcpy(dest, src); // copy src to dest, overwriting
  int len = strlen(str); // returns length of string
\end{lstlisting}

\end{document}
